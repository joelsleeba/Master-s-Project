% Copyright 2020 by Junwei Wang <i.junwei.wang@gmail.com>
%
% This file may be distributed and/or modified under the
% conditions of the LaTeX Project Public License, either version 1.3c
% of this license or (at your option) any later version.
% The latest version of this license is in
%   http://www.latex-project.org/lppl.txt
%   Compile this in xelatex

\documentclass[compress]{beamer}

\usepackage[english]{babel}
\usepackage{metalogo}
\usepackage{listings}
\usepackage{fontspec}
\usepackage{amsmath}
\usepackage{amsfonts}
\usepackage{amssymb}
\usepackage{mathrsfs}
\usepackage{tikz}

\usetheme{Nord}
% \usetheme[style=light]{Nord}

\newcommand{\R}{\mathbb R}
\newcommand{\N}{\mathbb N}
\newcommand{\Z}{\mathbb Z}
\newcommand{\F}{\mathbb F}
\newcommand{\Q}{\mathbb Q}
\newcommand{\C}{\mathbb C}
\newcommand{\T}{\mathbb T}

% \setsansfont{NotoSans Nerd Font}
% \setmonofont{FiraMono Nerd Font Mono}

\AtBeginSection[]
{
  \begin{frame}[c,noframenumbering,plain]
    \tableofcontents[sectionstyle=show/hide,subsectionstyle=show/show/hide]
  \end{frame}
}

\AtBeginSubsection[]
{
  \begin{frame}[c,noframenumbering,plain]
    \tableofcontents[sectionstyle=show/hide,subsectionstyle=show/shaded/hide]
  \end{frame}
}

\title{A Study in Fourier Analysis}
\subtitle{From circle, through the line, to the complex}
\author{Joel Sleeba}
\institute{IISER Thiruvananthapuram}
\date{\today}

\begin{document}

\begin{frame}[plain,noframenumbering]
  \maketitle
\end{frame}


\section{Fourier Series}

\begin{frame}{Structure and Topology of $\mathbb{T}$}{}
  \begin{itemize}
    \item Defining $\mathbb{T}$ as the set of equivalence class of the relation $x\sim y \iff x-y \in \mathbb{Z}$ and identifying classes in $\mathbb{T}$ with their representative element in $[0,1)$ as $[x] \to \{x\}$, where $\{x\}$ is the fractional part of $x$.
    \item Endow $\mathbb{T}$ with quotient topology by the map $f: \mathbb{R} \to \mathbb{T}:= x \to [x]$
    \item Lebesgue measure on $\mathbb{T}$ is defined by the Lebesgue measure of its identification in $[0,1)$.
  \end{itemize}
\end{frame}

\begin{frame}{Functions in $\T$}
  \begin{itemize}
    \item Functions in $\T$ are identified with periodic functions in $\R$ with period $1$ this again can be completely characterized by their values in $[0,1)$.
    \item By the quotient topology in $\T$, we see that continuous functions in $\T$ can identified with continuous functions in $\R$ with period $1$.
    \item Also by the Lebesgue measure defined on $\T$, we say $f \in L^p(\T)$ if the corresponding function in $[0, 1)$ is in $L^p[0, 1)$.
    \item For any two function $f, g \in L^1(\T)$, their convolution, $(f*g)(x)$ as $$(f*g)(x) = \int_0^1 f(x-y)g(y) \ dy$$
      is again in $L^1(\T)$
  \end{itemize}
\end{frame}

\begin{frame}{Fourier Coefficients}
  \begin{itemize}
    \item For $f \in L^1(\T)$, and $n \in \Z$ we define the $n^{th}$ Fourier coefficient of $f$ as $$\hat{f}(n) = \int_0^1 f(x)e^{-2\pi i n x} \ dx $$
    \item Also the Fourier series of $f \in L^1(\T)$ is defined as $$\sum_{n=-\infty }^\infty \hat{f}(n) e^{2\pi in x} $$
    \item Since we are interested in the convergence of the Fourier series, we will define the symmetric and Ces\`aro partial sums of the Fourier seres respectively as $$ S_N(x) = \sum_{n=-N}^N \hat{f}(n)e^{2\pi inx} \ \text{ and } \  \sigma_N(x) = \frac{1}{N}\sum_{n=0}^{N-1} S_n(x)$$
   \end{itemize}
\end{frame}

\begin{frame}{Summability Kernel}
  \begin{itemize}
    \item A collection of functions $K_N \in L^1(\T)$ are called a summability kernel if it satisfies the following properites
      \begin{enumerate}
        \item $\int_{0}^{1}K_N(x)dx = 1$
        \item $\int_{0}^{1}\lvert K_N(x)\rvert dx \le C$ for some constant $C>0$
        \item $\lim_{N\to\infty} \int_{\delta}^{1-\delta}\lvert K_N(x)\rvert dx= 0$
      \end{enumerate}
    \item We prove that if $K_N$ is a summability kernel in $L^1(T)$, then $(f*K_N)(x)$ converge to $f(x)$ in $L^1(\T)$. That is $$\int_0^1 \lvert f(x) - (f*K_N)(x) \rvert \ dx$$
  \end{itemize}
\end{frame}

\begin{frame}{Fej\'er Kernel and Ces\`aro Convergence}
  \begin{itemize}
    \item Fej\'er kernel defined as $$\Delta_N(x) = \sum_{n=-N}^N\left(1 - \frac{|n|}{N} \right)e^{2\pi inx}$$
      is a summability kernel we get that $(f*\Delta_N)$ converge to $f$ in $L^1(\T)$
    \item Moreover we see that $(f*\Delta_N)(x) = \sigma_N(x)$ and therefore the Ces\`aro partial sums of the Fourier series of $f$ converge to $f$ in $L^1(\T)$.
  \end{itemize}
\end{frame}

\begin{frame}{Fourier Series in $L^2(\T)$}
  \begin{itemize}
    \item Since $\T$ is identified with the finite measure space $[0, 1)$, we get that $L^2(\T) \subset L^1(\T)$. Theorefore the Fourier coefficients and series can be defined the same way as in $L^1(\T)$.
    \item Moreover we see that if $f \in L^2(\R)$, since the Fej\'er kernel, $\Delta_N \in L^\infty(\T)$, its Ces\`aro partial sum, $\sigma_N = (f*\Delta_N) \in L^2(\T)$
    \item As in $L^1(\T)$, we get that the Ces\`aro partial sums $\sigma_N$ converge to $f$ in $L^2(\T)$. That is $$\lim_{N\to \infty} \int_0^1 \left| f(x) - \sigma_N(x)\right| \ dx  = 0 $$
    \item The same results follow for functions in $L^p(\T)$
  \end{itemize}
\end{frame}

\begin{frame}{Fej\'er Theorem and Pointwise Convergence}
  Write if needed
\end{frame}

\section{Fourier Transforms in $\R$}

\begin{frame}{Fourier transforms in $L^1(\R)$}
  \begin{itemize}
    \item For any $f\in L^1(\R)$, the Fourier transform of $f$ is defined as $$\hat{f}(t) = \int_{-\infty}^{\infty} f(x)e^{-2\pi i t x} \ dx $$
    \item $\hat{f}$ is uniformly continuous and $$\lim_{|t| \to \infty} \hat{f}(t) = 0$$
  \end{itemize}
\end{frame}

\begin{frame}{Riemann Lebesgue Lemma}
\end{frame}

\begin{frame}{Fourier Inversion}
  \begin{itemize}
    \item Let $f \in L^1(\R)$, then the inverse Fourier transform is defined as $$\check{f}(t) = \int_{-\infty}^\infty f(x)e^{2\pi itx} \ dx$$
    \item We see that if $f \in L^1(\R)$, continuous at $x \in \R$ and its Fourier transform $\hat{f} \in L^1(\R)$, then $$\check{\hat{f}}(x) = f(x)$$
    \item Generalizing further we get that that if $f, \hat{f} \in L^1(\R)$ then $$\check{\hat{f}} \ \stackrel{a.e}{=} \ f$$
  \end{itemize}
\end{frame}

\begin{frame}{Fourier transforms in $L^2(\R)$}
  \begin{itemize}
    \item We consider the space $L^1(\R) \cap L^2(\R)$. Since it is a subspace of $L^1(\R)$, the definition of Fourier transform and inverse transform holds good in the smaller space.
    \item (Plancherel's Theorem) If $f \in L^1(\R) \cap L^2(\R)$, then $$\int_\R |f(x)|^2 \ dx = \int_\R |\hat{f}(t)|^2 \ dt$$
    \item Now since the collection of compactly supported functions in $\R$, $C_c(\R)$ is dense in $L^p(\R)$ for all $1 \le p < \infty$, we get that $L^1(\R) \cap L^2(\R)$ is dense in $L^2(\R)$.
    \item Moreover we see that the Fourier transform of functions in $L^1(\R) \cap L^2(\R)$ form a dense subset in $L^2(\R)$. Plancherel's theorem asserts that Fourier transform in $L^1(\R) \cap L^2(\R)$ is an isometry therefore we can extend Fourier transform to an isometry in $L^2(\R)$.
  \end{itemize}
\end{frame}

\section{Homolorphic Fourier Transforms}
\begin{frame}{Extending Domain to $\C$}
  \begin{itemize}
    \item Fourier transform of certain functions can be extended into a holomorphic functions in certain regions. That is for $z \in \C$, $$\hat{f}(z) = \int_\R f(x)e^{-2\pi izx} \ dx $$
      will be holomorphic in certain regions in $\C$.
    \item For example if $f(x) = e^{-|x|}$, then its Fourier transform, $\hat{f}(t) = \frac{1}{1+(2\pi t)^2}$ can be extended into holomorphic function in regions in the complex plane without the points $\pm \frac{i}{2\pi}$.
    \item We will focus on two types of functions in $L^2(\R)$
      \begin{enumerate}
        \item $f(x) = 0, (x<0)$
        \item $f(x) = 0, (x \notin (-A, A))$
      \end{enumerate}
  \end{itemize}
\end{frame}

\begin{frame}{Paley Wiener Theorem 1}
  \begin{itemize}
    \item Let $F \in L^2(\R)$ such that $F(x) = 0$ in $(-\infty, 0)$. Then $f: \Pi^+ \to \C$ defined as, $$f(z) = \int_0^\infty F(x) e^{2\pi it z} \ dz$$
      is in $H(\Pi^+)$ if and only if $$\sup_{0<y<\infty} \frac{1}{2\pi}\int_\R f(x+iy) \ dx = \int_0^\infty |F(x)|^2 \ dx < \infty$$
  \end{itemize}
\end{frame}

\begin{frame}{Paley Wiener Theorem 2}
    The following statements are equivalent:
      \begin{enumerate}
        \item $f: \C \to \C$ is an entire function satisfying $|f(z)| \le Ce^{2\pi A |z|}$ and $$\int_{-\infty}^\infty |f(x+iy)|^2 \ dx < \infty$$
        \item There exist an $F \in L^2(\R)$ is compactly supported in $[-A, A]$ such that $$f(z) = \int_{-A}^A F(x) e^{2\pi iz x} \ dx$$
      \end{enumerate}
\end{frame}

\begin{frame}{Consequence of Paley Wiener Theorems}
\end{frame}


\section{Future Directions}

\begin{frame}{Schwartz Class}
  \begin{itemize}
    \item A smooth function $f:\mathbb{R}^n \to \mathbb{C}$, $f$ is called a \emph{Schwartz function} if for any given multi index $\alpha, \beta$, there exists a positive constant $C_{\alpha, \beta}$ such that $$\rho_{\alpha, \beta} = \sup_{x \in \mathbb{R}^n} \left|x^\alpha (D^\beta f)x \right| = C_{\alpha, \beta} < \infty$$
    \item Here $\rho_{\alpha, \beta}(f)$ is called \emph{Schwartz seminorm of $f$}. The collection of all such functions is called the \emph{Schwartz space} of $\mathbb{R}^n$ and is denoted by $\mathscr{S}(\mathbb{R}^n)$.
    \item Schwartz class is dense in $L^p(\R^n)$ for all $1\le p < \infty$.
  \end{itemize}
\end{frame}

\begin{frame}{Fourier transforms in $R^n$}
  \begin{itemize}
    \item Fourier transform of $f \in \mathscr{S}(\R^n)$, $\hat{f} : \R^n \to \C^n$ is defined as $$\hat{f}( \textbf{t} ) = \int_{\R^n} f(x)e^{-2 \pi i \textbf{t} \cdot \textbf{x}} \ dx$$
    \item Parseval's identity holds in Schwartz class $$\|\hat{f}\|_2 = \|f\|_2$$
    \item Fourier transform is a homeomorphism in $\mathscr{S}(\R^n)$.
    \item By Parseval's identity Fourier transform can be extended into whole of $\R^n$
  \end{itemize}
\end{frame}

\begin{frame}{Restriction Conjecture}
  
\end{frame}

\end{document}

%%% Local Variables:
%%% mode: latex
%%% TeX-master: t
%%% TeX-engine: xetex
%%% End:
