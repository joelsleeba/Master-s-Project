%%% the following commands uses the 'mdframed' package
%%% to give color boxes around theorems, definitions, 
%%% and proposition. It also takes care of splitting
%%% the numbering of theorem and derieved classes from
%%% theorem. If you don't want color boxes, comment
%%% the block below and uncomment the next block.
%%% I thought you'll need another package called 
%%% 'aliascnt' for the next line, but boy! my code
%%% is too good.

% \newtheoremstyle{mytheoremstyle}{3pt}{3pt}{\normalfont}{0cm}{\rmfamily\bfseries}{}{1em}{{\color{black}\thmname{#1}~\thmnumber{#2}}\thmnote{\,--\,#3}}
% \newtheoremstyle{myproblemstyle}{3pt}{3pt}{\normalfont}{0cm}{\rmfamily\bfseries}{}{1em}{{\color{black}\thmname{#1}~\thmnumber{#2}}\thmnote{\,--\,#3}}
%
% \theoremstyle{mytheoremstyle}
% \newmdtheoremenv[linewidth=1pt,backgroundcolor=shallowGreen,linecolor=deepGreen,leftmargin=0pt,innerleftmargin=20pt,innerrightmargin=20pt,]{theorem}{Theorem}[section]
% \newmdtheoremenv[linewidth=1pt,backgroundcolor=mygray,linecolor=deepGreen,leftmargin=0pt,innerleftmargin=20pt,innerrightmargin=20pt,]{corollary}{Corollary}[section]
% \newmdtheoremenv[linewidth=1pt,backgroundcolor=mygray,linecolor=ocre,leftmargin=0pt,innerleftmargin=20pt,innerrightmargin=20pt,]{lemma}{Lemma}[section]
% \newmdtheoremenv[linewidth=1pt,backgroundcolor=mygray,linecolor=ocre,leftmargin=0pt,innerleftmargin=20pt,innerrightmargin=20pt,]{proposition}{Proposition}[section]
% \theoremstyle{mytheoremstyle}
% \newmdtheoremenv[linewidth=1pt,backgroundcolor=shallowBlue,linecolor=deepBlue,leftmargin=0pt,innerleftmargin=20pt,innerrightmargin=20pt,]{definition}{Definition}[section]
%
% \theoremstyle{myproblemstyle}
% \newmdtheoremenv[linecolor=black,leftmargin=0pt,innerleftmargin=10pt,innerrightmargin=10pt,]{problem}{Problem}[section]

%%% Standard theorem numbering styles. 
%%% \newtheorem{newtype}{Display Label}[reset counter at every]
%%% Here everything except corollary will reset their numbering
%%% at every new section. That is if the last theorem 
%%% in section 2 of chapter 1 is labelled 1.2.5 Then the next
%%% theorem at section 3 of the same chapter will be 1.3.1. 
\theoremstyle{plain} % default; italic text, extra space above and below
\newtheorem{theorem}{Theorem}[section]
\newtheorem{proposition}{Proposition}[section]
\newtheorem{lemma}{Lemma}[section]
\newtheorem{corollary}{Corollary}[theorem]
\newtheorem{problem}{Problem}[section]

\theoremstyle{definition} % upright text, extra space above and below
\newtheorem{definition}{Definition}[section]

\theoremstyle{remark} % upright text, no extra space above or below
\newtheorem*{note}{Note} %'Notes' in italics and without counter 

%%% Use this format to inherit the same numbering for all of
%%% the below items. They'll all be numbered with the counter
%%% of theorem. That is lemma after theorem 1.4.5 will be 
%%% 1.4.6
% \newtheorem{claim}[theorem]{Claim}
% \newtheorem{fact}[theorem]{Fact}
% \newtheorem{notation}[theorem]{Notation}
% \newtheorem{observation}[theorem]{Observation}
% \newtheorem{conjecture}[theorem]{Conjecture}
