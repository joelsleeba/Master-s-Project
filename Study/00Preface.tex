% TeX root = main.tex
\section{Preface}
I plan to write and detail everything(almost) I study and learn for my Master's project into these latex files. I assume it will be much easier to track whatever I have learned and to have a good overview of the topic with this note taking. Also since I am doing it on \LaTeX\ I am sure it will save me from the last minute rush to type everything out and make the report of the project. I will start with Fourier series and will introduce new concepts when they are required as we go along. So, let us start


Joel Sleeba\\
\today

\newpage

\section{Preliminaries}

  \subsection{Measure theory}

  \begin{definition}[$\sigma$-algebra]
    \label{def:sigma_algrebra}
    A collection $\Sigma$ of subsets of a set $X$ is called a $\sigma$-algebra if it satisfy the following properties
    \begin{enumerate}
      \item $X\in \Sigma$
      \item If $E \in \Sigma$, then $E^c \in \Sigma$.
    \item If $E_1, E_2, \dots$ are elements of $\Sigma$, then $\bigcup_{i=1}^\infty E_i \in \Sigma$.
    \end{enumerate}
  \end{definition}
