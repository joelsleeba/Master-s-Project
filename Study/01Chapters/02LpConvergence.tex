% TeX root = main.tex
\chapter{Fourier Series in $L^p(\T)$}
Before we go into the general $L^p$, we'll first recall some importatnt results from functional analysis which are essential for the discussion of further topics.

Recall that in \autoref{def:Lp_function}, we've discussed what is an $L^p$ function for a general space $S$. Here the space is $\T$ and $L^p(\T)$ are precisely the set of all $L^p$ functions in $\T$.

\begin{theorem}[Holder's Inequality]
   \label{thm:Holder_inequality}
  Let $1 \le  p \le \infty$ and $q$ such that $1/p + 1/q = 1$ (for convention we will assume that the tuples $(1, \infty)$, and $(\infty, 1)$, satify the above relation) then for lebesgue measure space $S$ and functions $f \in L^p(S)$, and $g \in L^q(S)$
  $$ \left| \int_S f(x)g(x) dx \right| \le \|f\|_p \|g\|_q$$
  where $$\|f\|_p = \left(\int_s |f(x)|^p \right)^{\frac{1}{p}}$$ as in \autoref{def:Lp_function}.
\end{theorem}

\begin{theorem}[Minkowski's Inequality]
  \label{thm:Minkowski_inequality}
  Let $1 \le p \le \infty$ then for a lebesgue measure space $S$ and a fucntion $f \in L^p(S)$, then 
  $$\|f+g\|_p \le \|f\|_p + \|g\|_p$$
\end{theorem}

The proofs of the above two theorems can be found in any measure theory textbook and since it is a common proof, we'll ommit it from detailing it here.

\section{Fourier Series in $L^2(\T)$}
Before we proceed with the fourier coefficients and fourier series for $L^2(\T)$ functions we first prove some important results.
\begin{proposition}
  \label{prop:L2_functions_are_continuous_in_L2_norm}
  If $f \in L^2(\T)$, then
  \begin{displaymath}
   \lim_{\delta \to 0} \int_0^1 |f(x+\delta) - f(x)|^2 dx = 0
  \end{displaymath}
\end{proposition}
\begin{proof}
  From measure theory we know that continuous functions are dense in $L^2(\T)$. Then given any $\epsilon$ there exists a function $g\in C(\T)$ such that $\|f-g\|_2 < \epsilon$. Then,
  $$ f(x+\delta) - f(x) = (f(x+\delta) - g(x+\delta)) - (f(x) - g(x)) + ( g(x+\delta) - g(x))$$
  Now by triangle inequality, (i.e minkowski's inequality for $p = 2$), 
  $$ \|f(x+\delta) - f(x)\|_2 = \|f(x+\delta) - g(x+\delta)\|_2 - \|f(x) - g(x)\|_2 + \|g(x+\delta) - g(x)\|_2$$
  Now since $g$ is continuous on $\T$ it is uniformly continuous (since it is continuous on $\R$, it'll be continuous on $[0,1]$, a compact set) and therefore $\delta$ can be taken such that $\|g(x+\delta) - g(x) \|_2 < \epsilon$. Hence the theorem.
\end{proof}

\begin{proposition}
  \label{prop:convolution_of_L2_functions}
  If $f, g \in L^2(\T)$, then their convolution, $f*g$ is continuous and moreover $\|f*g\|_{\infty} \le \|f\|_2 \|g\|_2$
\end{proposition}
\begin{proof}
  $$(f*g)(x+\delta) - (f*g)(x) = \int_0^1 f(u)(g(x+\delta - u) - g(x-u)) du $$
  Therefore by Cauchy Shwarz inequality, 
  \begin{align*}
    |f*g(x+\delta) - f*g(x)| &= \left| \int_0^1 f(u) (g(x + \delta -u)- g(x-u)) du \right| \\
          &\le \|f\|_2 \left( \int_0^1|g(x+\delta - u) - g(x-u)|^2 du \right)^{1/2} \\
  \end{align*}
  which converge to zero as $\delta \to 0$ by \autoref{prop:L2_functions_are_continuous_in_L2_norm}. Therefore $f*g$ is continuous in $\T$.
  Also, 
  \begin{align*}
    |f*g(x)| &= \left| \int_0^1 f(u)g(x-u) du \right| \\
            &\le \left(\int_0^1 |f(u)|^2 du \right)^{1/2} \left(\int_0^1 |g(u)|^2 du \right)^{1/2} \\
            &= \|f\|_2 \|g\|_2
  \end{align*}
  Hence the theorem.
\end{proof}

 Now that we've proved some important results, we'll define the fourier coeffients $\hatf(n)$ the same way we defined them for $\L^1(\T)$ in \autoref{def:fourier_coefficient} as 
$$\hat{f}(n) = \int_0^1 f(x)e^{-2\pi inx} dx$$

Note that $\hatf(n)$ is a finite quantity since $|e^{ix}| = 1$ and,
$$ |\hatf(n)| = \left| \int_0^1 f(x)e^{-2\pi inx} dx \right| \le \|f\|_2\left(\int_0^1 |e^{-2\pi inx}|^2 dx\right)^{1/2} = \|f\|_2 $$

Hence we define the fourier series the same way as in \autoref{def:fourier_series}. Then we'll investigate if the Fourier series of $L^2(\T)$ functions are Ces\'aro summable to $f$. In fact this is true.

\begin{theorem}[Fourier series of $L^2(\T)$ functions are Ces\'aro summable]
  If $f\in L^2(\T)$, and $\sigma_N(x)$ is the $N^{th}$ Ces\'aro partial sum of the Fourier series of $f$, then
  $$ \lim_{N \to \infty} \int_0^1 |f(x) - \sigma_N(x)|^2 \ dx = 0$$
\end{theorem}

\begin{proof}
  From \autoref{prop:properties_of_fejer_kernel} we know that
  \begin{displaymath}
    \sigma_N(x) = \int_0^1 \Delta_N(u)f(x-u) \ du 
  \end{displaymath}
  Hence,
  $$ f(x) - \sigma_N(x) = \int_0^1 (f(x) - f(x-u))\Delta_N(u) \ du $$ 
  By Holder's inequality in \autoref{thm:Holder_inequality} applied to $(f(x) - f(x-u))\sqrt{\Delta_N(u)}$ and $\sqrt{\Delta_N(u)}$ 
  \begin{align*}
    |f(x) - \sigma_N(x)| &= \left| \int_0^1 (f(x) - f(x-u))\Delta_N(u) \right| \ du \\
          &\le \left( \int_0^1 |f(x) - f(x-u)|^{2} \Delta_N(u) \ du \right)^{1/2} \left( \int_0^1 \Delta_N(u) \ du \right)^{1/2} \\
  \end{align*}
  Since by \autoref{prop:properties_of_fejer_kernel}, $\int_0^1 \Delta_N(x) dx = 1$ and by Tonelli's theorem, 
  \begin{align*}
    \int_0^1 |f(x) - \sigma_N(x)|^2 dx &= \int_0^1 \int_0^1 |f(x) - f(x-u)|^{2} \Delta_N(u) \ du  \ dx \\
          & = \int_0^1 \Delta_N(u) \int_0^1 |f(x) - f(x-u)|^{2} \ dx  \ du \\
          & = \int_{-\delta}^\delta + \int_\delta^{1-\delta} \\
          & = I_1 + I_2
  \end{align*}

  Also from \autoref{prop:L2_functions_are_continuous_in_L2_norm}, given $\epsilon$, we can find $\delta$ such that 
  $$ \int_0^1|f(x) - f(x-\delta)|^2 \ dx < \epsilon$$

  Then for that choice of $\delta$,
  $$|I_1| \le \epsilon \int_{-\delta}^\delta \Delta_N(u) du \le \epsilon \int_0^1 \Delta_N(u) du = \epsilon$$
  To prove $I_2$ is also bounded, we'll use a small trick aided by Minkowski's inequality. We know that $\|f-g\|_p \le \|f\|_p + \|g\|_p$. Therefore $\|f-g\|_p \le 2\max\{\|f\|_p, \|g\|_p\}$ and $\|f-g\|_p^p \le 2\max\{\|f\|_p^p, \|g\|_p^p\}$, and finally, $\|f-g\|_p^p \le 2(\|f\|_p^p + \|g\|_p^p)$.Then for $p=2$, 
  $$ \int_0^1 |f(x) - f(x-u)|^2 dx \le 2(\|f\|_2^2 + \|f\|_2^2) = 4\|f\|_2^2$$
  Therefore by \autoref{prop:properties_of_fejer_kernel}, we get 
  $$|I_2| = \int_\delta^{1-\delta} \Delta_N(u) \int_0^1 |f(x) - f(x-u)|^{2} \ dx  \ du \le 4\|f\|_2^2 \int_\delta^{1-\delta}\Delta_N(u) du$$
  But by \autoref{prop:properties_of_fejer_kernel} 
  $$ \int_\delta^{1-\delta}\Delta_N(u)du = 2\int_\delta^{1/2}\Delta_N(x) \le \frac{1}{2N\delta}$$
  Which implies, 
  $$ |I_2| \le \frac{2\|f\|_2^2}{N\delta}$$
  Therefore $I_2$ converge to $0$ as $N \to \infty$. Hence the theorem.
\end{proof}

\section{Fourier Series in $L^p(\T)$}
\begin{proposition}
  \label{prop:Lp_functions_are_continuous_in_Lp_norm}
  If $f \in L^p(\T)$, then
  \begin{displaymath}
   \lim_{\delta \to 0} \int_0^1 |f(x+\delta) - f(x)|^p dx = 0
  \end{displaymath}
\end{proposition}
\begin{proof}
  From measure theory we know that continuous functions are dense in $L^p(\T)$. Then given any $\epsilon$ there exists a function $g\in C(\T)$ such that $\|f-g\|_p < \epsilon$. Then,
  $$ f(x+\delta) - f(x) = (f(x+\delta) - g(x+\delta)) - (f(x) - g(x)) + ( g(x+\delta) - g(x))$$
  Now by triangle inequality, (i.e minkowski's inequality for $p = 2$), 
  $$ \|f(x+\delta) - f(x)\|_p = \|f(x+\delta) - g(x+\delta)\|_p - \|f(x) - g(x)\|_p + \|g(x+\delta) - g(x)\|_p$$

  Now since $g$ is continuous on $\T$ it is uniformly continuous (since it is continuous on $\R$, it'll be continuous on $[0,1]$, a compact set) and therefore $\delta$ can be taken such that $\|g(x+\delta) - g(x) \|_p < \epsilon$. Hence the theorem.
\end{proof}

\begin{proposition}
  \label{prop:convolution_of_Lp_functions}
  If $f \in L^p(\T)$ and $g \in L^q(\T)$, then their convolution, $f*g$ is continuous and moreover $\|f*g\|_{\infty} \le \|f\|_p \|g\|_q$
\end{proposition}
\begin{proof}
  $$(f*g)(x+\delta) - (f*g)(x) = \int_0^1 f(u)(g(x+\delta - u) - g(x-u)) du $$
  Therefore by Minkowski's inequality from \autoref{thm:Minkowski_inequality}, 
  \begin{align*}
    |f*g(x+\delta) - f*g(x)| &= \left| \int_0^1 f(u) (g(x + \delta -u)- g(x-u)) du \right| \\
          &\le \|f\|_p \left( \int_0^1|g(x+\delta - u) - g(x-u)|^q du \right)^{1/q} \\
  \end{align*}
  which converge to zero as $\delta \to 0$ by \autoref{prop:Lp_functions_are_continuous_in_Lp_norm}. Therefore $f*g$ is continuous in $\T$.
  Also, 
  \begin{align*}
    |f*g(x)| &= \left| \int_0^1 f(u)g(x-u) du \right| \\
            &\le \left(\int_0^1 |f(u)|^p du \right)^{1/p} \left(\int_0^1 |g(u)|^q du \right)^{1/q} \\
            &= \|f\|_p \|g\|_q
  \end{align*}
  Hence the theorem.
\end{proof}


Now that we've proved some important results, we'll define the fourier coeffients $\hatf(n)$ the same way we defined them for $\L^1(\T)$ in \autoref{def:fourier_coefficient} as 
$$\hat{f}(n) = \int_0^1 f(x)e^{-2\pi inx} dx$$

Note that $\hatf(n)$ is a finite quantity since $|e^{ix}| = 1$ and,
$$ |\hatf(n)| = \left| \int_0^1 f(x)e^{-2\pi inx} dx \right| \le \|f\|_p\left(\int_0^1 |e^{-2\pi inx}|^q dx\right)^{1/q} = \|f\|_p $$ where $1/p + 1/q = 1$.

Hence we define the fourier series the same way as in \autoref{def:fourier_series}. Then we'll investigate if the Fourier series of $L^p(\T)$ functions are Ces\'aro summable to $f$. In fact this is true.

\begin{theorem}[Fourier series of $L^p(\T)$ functions are Ces\'aro summable]
  If $f\in L^p(\T)$, and $\sigma_N(x)$ is the $N^{th}$ Ces\'aro partial sum of the Fourier series of $f$, then
  $$ \lim_{N \to \infty} \int_0^1 |f(x) - \sigma_N(x)|^p \ dx = 0$$
\end{theorem}

\begin{proof}
  From \autoref{prop:properties_of_fejer_kernel} we know that
  \begin{displaymath}
    \sigma_N(x) = \int_0^1 \Delta_N(u)f(x-u) \ du 
  \end{displaymath}
  Hence,
  $$ f(x) - \sigma_N(x) = \int_0^1 (f(x) - f(x-u))\Delta_N(u) \ du $$ 
  By Holder's inequality as in \autoref{thm:Holder_inequality} applied to $(f(x) - f(x-u))(\Delta_N(u))^{1/p}$ and $(\Delta_N(u))^{1/q}$ 
  \begin{align*}
    |f(x) - \sigma_N(x)| &= \left| \int_0^1 (f(x) - f(x-u))\Delta_N(u) \right| \ du \\
          &\le \left( \int_0^1 |f(x) - f(x-u)|^{p} \Delta_N(u) \ du \right)^{1/p} \left( \int_0^1 \Delta_N(u) \ du \right)^{1/q} \\
  \end{align*}
  Since by \autoref{prop:properties_of_fejer_kernel}, $\int_0^1 \Delta_N(x) dx = 1$ and by Tonelli's theorem, 
  \begin{align*}
    \int_0^1 |f(x) - \sigma_N(x)|^2 dx &= \int_0^1 \int_0^1 |f(x) - f(x-u)|^{p} \Delta_N(u) \ du  \ dx \\
          & = \int_0^1 \Delta_N(u) \int_0^1 |f(x) - f(x-u)|^{p} \ dx  \ du \\
          & = \int_{-\delta}^\delta + \int_\delta^{1-\delta} \\
          & = I_1 + I_2
  \end{align*}

  Also from \autoref{prop:Lp_functions_are_continuous_in_Lp_norm}, given $\epsilon$, we can find $\delta$ such that 
  $$ \int_0^1|f(x) - f(x-\delta)|^p \ dx < \epsilon$$

  Then for that choice of $\delta$,
  $$|I_1| \le \epsilon \int_{-\delta}^\delta \Delta_N(u) du \le \epsilon \int_0^1 \Delta_N(u) du = \epsilon$$
 To prove $I_2$ is also bounded, we'll use a small trick aided by Minkowski's inequality. We know that $\|f-g\|_p \le \|f\|_p + \|g\|_p$. Therefore $\|f-g\|_p \le 2\max\{\|f\|_p, \|g\|_p\}$ and $\|f-g\|_p^p \le 2\max\{\|f\|_p^p, \|g\|_p^p\}$, and finally, $\|f-g\|_p^p \le 2(\|f\|_p^p + \|g\|_p^p)$.Then, 
  $$ \int_0^1 |f(x) - f(x-u)|^p dx \le 2(\|f\|_p^p + \|f\|_p^p) = 4\|f\|_p^p$$
  Therefore by \autoref{prop:properties_of_fejer_kernel}, we get 
  $$|I_2| = \int_\delta^{1-\delta} \Delta_N(u) \int_0^1 |f(x) - f(x-u)|^{p} \ dx  \ du \le 4\|f\|_p^p \int_\delta^{1-\delta}\Delta_N(u) du$$
  But by \autoref{prop:properties_of_fejer_kernel} 
  $$ \int_\delta^{1-\delta}\Delta_N(u)du = 2\int_\delta^{1/2}\Delta_N(x) \le \frac{1}{2N\delta}$$
  Which implies, 
  $$ |I_2| \le \frac{2\|f\|_p^p}{N\delta}$$
  Therefore $I_2$ converge to $0$ as $N \to \infty$. Hence the theorem.
\end{proof}

If looked close enough one can see that the proof of convergence of fourier series in $L^p(\T)$ is almost the same as in $L^2(\T)$. This is in fact true and $L^2$ convergence is just a special case of $L^p$ convergence.
