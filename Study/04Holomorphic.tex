%TeX root = main.tex
\newpage
\section{Holomorphic Fourier Transforms}

\subsection{Motivation and Basic Results}
In last chapter we explored the Fourier transforms of functions in $\R$. Frequently the fourier transform of functions in $\R$ can be extended to a holomorphic function in certain regions. For example say $f(x) = e^{-|x|}$, then $\hatf(t) = \frac{1}{1+(2\pi t)^2}$. 
Since, 
\begin{align*}
  \hatf(t) = \int_\R e^{-|x|}e^{-2 \pi itx} \ dx &= \int_{-\infty}^0 e^{x(1-2\pi it)} \ dx + \int_0^{\infty} e^{-x(1+2\pi it)} \ dx \\
  & = \left[ \frac{e^{x(1-2\pi it)}}{1-2\pi it}\right]_{-\infty}^0 - \left[ \frac{e^{-x(1+2\pi it)}}{1+2\pi it}\right]^{\infty}_0 \\
  & = \frac{1}{1-2\pi it} + \frac{1}{1+2\pi it} \\
  &= \frac{1}{1+(2\pi t)^2}
\end{align*}
Clearly $w(z) = \frac{1}{1+(2\pi z)^2}$ is a holomorphic extension of $\hatf$ into the regions of the complex plane without the points $z=\pm \frac{i}{2\pi}$.

We will study two classes of functions which can be extended in this manner. The first class of such functions is $$f(z) = \int_0^\infty F(t)e^{2\pi i tz}\ dt$$
where $z\in \Pi^+ = \{z\in \C \ | \Im(z) > 0 \}$ and $F \in L^2(\R)$ is a function which vanishes on $(-\infty, 0)$. The second class of functions will be $$f(z) = \int_{-A}^A F(t)e^{2\pi i t z} \ dt$$
where $0<A<\infty$ and $F \in L^2(-A, A)$. We'll now prove some important results regarding these two classes of functions

\begin{proposition}
  Let $F\in L^2(\R)$ such that $F$ vanishes at $(-\infty , 0)$. Then $f : \Pi^+ \to \C$ defined as, $$f(z) = \int_0^\infty F(t) e^{2\pi itz} \ dt$$
  is holomorphic in the upper half plane, i.e $f \in H(\Pi^+)$. Moreover $f$ restricted to horizontal lines in $\Pi^+$ is bounded in the  $L^2$ norm and the bound is independent of the horizontal line. That is if $z = x+iy$, then $$\sup_{0<y<\infty}\int_{-\infty}^\infty \left|f(x + iy)\right|^2 \ dx < \infty$$
\end{proposition}
\begin{proof}
  Let $z \in \Pi^+$. Then there exists a $\delta$ such that $0< \delta < \Im(z)$. Since $\Pi^+$ is open, there exists a sequence $z_n$ in $\Pi^+$ such that $\delta < \Im(z_n)$, and  $z_n \to z$. Also 
  \begin{align*}
    \left|e^{2 \pi itz_n} - e^{2\pi itz}\right|^2 &= \left|\left(e^{2\pi it z_n} - e^{2\pi itz}\right)^2\right| \\ 
    &= \left| e^{4\pi itz_n} + e^{4\pi itz} - 2e^{2\pi it(z+z_n)} \right| \\
    &\le \left|e^{4\pi itz_n}\right| + \left|e^{4\pi itz}\right| + 2\left|e^{2\pi it(z_n + z)}\right| \\
    &= e^{-4\pi t \Im(z_n)} + e^{-4 \pi t \Im(z)} + 2e^{-2\pi t \Im(z_n - z)} \\
    &\le e^{-4\pi t \delta} + e^{-4\pi t \delta} + 2e^{-4 \pi t\delta} \\
    &=4e^{-4\pi t \delta}
  \end{align*}
  Now since $$\int_0^\infty 4e^{-4\pi t\delta} \ dt = \frac{4}{2\pi \delta}$$
  the integrad is $L^1$ in $(0, \infty)$ and therefore by dominated convergence theorem, $$\lim_{n \to \infty} \int_0^\infty \left|e^{2\pi itz_n} - e^{2\pi itz}\right|^2 \ dt = 0$$
  Now therefore if $w \in \Pi^+$ and $w_n \to w$, then 
  \begin{align*}
    f(w_n) - f(w) &= \int_0^\infty F(t)(e^{2\pi it w_n} - e^{2\pi i tw}) \ dt \\
    & \le \|F\|_2 \|e^{2\pi itw_n} - e^{2\pi itw}\|_2
  \end{align*}
  By assumption $\|F\|_2$ is a finite quantity and from above result $\|e^{2\pi itw_n} - e^{2\pi itw}\|_2 \to 0$ as $w_n \to w$. Therefore $f(w_n) \to f(w)$. Since our choice of $w$ was arbitary, this implies that $f$ is continuous everywhere in $\Pi^+$. Also if $\gamma$ is any closed path in $\Pi^+$, then 
  \begin{align*}
    \int_\gamma f(z)\ dz &= \int_\gamma \int_0^\infty F(t) e^{2\pi i t z} \ dt \ dz \\
    & = \int_0^\infty F(t) \int_\gamma e^{2\pi itz} \ dz \ dt \\
    & = \int_0^\infty F(t) \cdot 0 \ dt \\
    & = 0
  \end{align*}
  Note that the change of integral is justified by Fubini's theorem. \textcolor{red}{Justify how!} Therefore by Morera's theorem, $f$ is holomorphic everywhere in the upper half plane.  

  Now consider $z=x+iy$ where $i = \sqrt{-1}$, then $f$ can be written as $$f(x+iy) = \int_0^\infty F(t)e^{2\pi i t (x+iy)}\ dt = \int_0^\infty F(t)e^{-2\pi ty}e^{2\pi i tx} \ dx$$ 
  Now for a fixed $y \in \R$, consider $g_y(t) = F(t)e^{-2\pi ty}$. Then from above equation $f(x+iy) = \widehat{g_y}(x)$. By Holder's inequality we get that 
  \begin{align*}
    \int_0^\infty |g_y(t)| \ dt &= \int_0^\infty \left|F(t)e^{-2\pi ty}\right| \ dt \\
    &\le \left(\int_0^\infty \left|F(t)\right|^2 \ dt \right)^{\frac{1}{2}} \left(\int_0^\infty e^{-4\pi ity} \ dt \right)^{\frac{1}{2}}
  \end{align*}
  where both of the integrals are finite. Therefore $g_y \in L^1(0, \infty)$. Again since $0<e^{-x}\le1$ when $0\le x<\infty$, we know that $$\int_0^\infty |g_y(t)|^2 \ dt = \int_0^\infty \left|F(t)\right|^2 \left| e^{-2\pi ty} \right|^2 \ dt \le \int_0^\infty \left|F(t)\right|^2 \ dt$$
  which is again finite. Therefore $g\in L^2(0, \infty)$. Therefore by Plancherel's theorem (refer theorem \ref{thm:Plancherel's_theorem}) on $g_y$ we get that, $$\int_{-\infty}^{\infty}\left|f(x+iy)\right|^2 \ dx = \int_{-\infty}^{\infty} \left|\widehat{g_y}(x)\right|^2 \ dx= \int_{-\infty}^{\infty} \left|g_y(t)\right|^2 \ dt = \int_0^\infty \left|g_y(t)\right|^2 \ dt $$
Note that the change in limit in the last integral is because $g_y(t) = 0$ when $t < 0$. But from last inequality $$\int_0^\infty \left|g_y(t)\right|^2 \ dt \le \int_0^\infty \left|F(t)\right|^2 \ dt$$
  which is finite since $F \in L^2(0, \infty)$, and independent of $y$. Therefore $$\sup_{0<y<\infty} \int_{-\infty}^\infty \left|f(x+iy)\right|^2 \ dx \le \int_0^\infty \left|F(t)\right|^2 \ dt $$
\end{proof}

\begin{proposition}
  Let $0<A<\infty$ and $F \in L^2(-A, A)$. Then $f:\C \to \C$ defined as $$f(z) = \int_{-A}^A F(z)e^{2\pi itz} \ dt$$
  is an entire function which satisfy $$|f(z)| \le Ce^{2 \pi A|z|}$$
  for some constant $C$ and $f$ when restricted to horizontal lines is bounded in the $L^2$ norm. That is if $f(z)$ is written as $f(x+iy)$ then  $$\int_{-\infty}^{\infty} f(x+iy) \ dx < \infty$$
\end{proposition}
\begin{proof}
  Like in proof of the above theorem for $z\in \C$ we write $z = x+iy$. Then $$f(x+iy) = \int_{-A}^A F(t) e^{2\pi it(x+iy)} \ dt = \int_{-A}^A F(t)e^{-2\pi ty}e^{2\pi itx} \ dt$$
  Therefore since $|e^{2\pi it x}| = 1$, $$\left|f(x+iy)\right| \le \int_{-A}^A \left|F(t)\right| e^{-2\pi ty} \ dt \le e^{2\pi A|y|} \int_{-A}^A \left|F(t)\right| \ dt $$
  We know that $F \in L^2(-A, A)$. But since $(-A, A)$ under Lebesgue measure is a finite measure space, $f \in L^1(-A, A)$. And therefore the last integral above is finite. Let's call it $C$. Also since $e^{\Im(z)} \le e^z$, it follows that $$\left|f(z) \right| \le Ce^{2\pi A |z|}$$

  Now consider points $z \in \C$. Then for $\delta \in \C$,
  \begin{align*}
    f(z+\delta) - f(z) &= \int_{-A}^A F(t)\left(e^{it(z+\delta)} - e^{itz}\right) \ dt \\
    &\le \left(\int_{-A}^A \left|F(t)\right|^2\ dt \right)^{\frac{1}{2}} \left(\int_{-A}^A \left|e^{it(z+\delta)}-e^{itz}\right|^2 \ dt \right)^{\frac{1}{2}}
  \end{align*}
  Now the first integral is finite since $F\in L^2(-A, A)$. Again since $e^{itz}$ is a continuous function in a finite measure space it is in $L^2(-A, A)$ and by a change of variable $r = At$, a we can use proposition \ref{prop:L2_functions_are_continuous_in_L2_norm} and conclude that the second integral converge to zero as $\delta \to 0$. Therefore $f$ is continuous in the whole of $\C$. 

  Moreover if $\gamma$ is any closed path in $\C$, and $\Omega$ is the interior of the space enclosed by $\gamma$, then $(-A, A)\times\Omega$ is a finite measure space. Therefore by Fubini's Theorem we change of order of integration and $$\int_\gamma f(z) \ dz = \int_\gamma \int_{-A}^A F(t)e^{2\pi itz} \ dt \ dz = \int_{-A}^A F(t) \int_\gamma e^{2\pi itz} \ dz \ dt = \int_{-A}^A F(t)\cdot 0 \ dt = 0$$
  since $e^{2\pi itz}$ is entire. Then by Morera's theorem, $f$ is entire.
\end{proof}

\subsection{Paley Wiener Theorems}
\
