% TeX root = ../main.tex

\chapter{Results from Measure Theory}
Here'll we'll discuss some important results from measure theory which are essential for our subject. We already defined what is an $L^1$ function in a given space at \autoref{def:Lp_function}.

\begin{theorem}
  \label{thm:continuous_functions_in_T_are_dense_in_L1}
  Continuous functions in $\T$, (refer \autoref{def:continuous_functions_on_T}) are dense in $L^p(\T)$ for $1\le p < \infty$.
\end{theorem}
\begin{proof}
  This is a direct consequence of \autocite[Theorem~3.14 on \pno~69]{papaRudin}. Since $\T$ is identified with $[0, 1)$, all continuous functions in $\T$ are compactly supported.
\end{proof}

\begin{theorem}
  \label{thm:L1_functions_are_dense_in_L2}
  $L^1(\R) \cap L^2(\R)$ is dense in $L^2(\R)$.
\end{theorem}
\begin{proof}
  Let $C_c(\R)$ denote the set of compactly supported continuous functions in $\R$. Since every function is continuous and compactly supported, $C_c(\R) \subset L^p(\R)$ for all $1\le p < \infty$. Therefore $C_c(\R) \subset L^1(\R) \cap L^2(\R)$. Then by \autocite[Theorem~3.14 on \pno~69]{papaRudin} $C_c(\R)$ is dense in $L^2(\R)$ and therefore $L^1(\R) \cap L^2(\R)$ is dense in $L^2(\R)$.
\end{proof}

If you follow the proof of the above theorem close enough, you'll see that we can make a stronger claim. Since, $C_c(\R) \subset L^p(\R)$ for all $1\le p < \infty$, $$C_c(\R) \subset \bigcap_{1\le p < \infty} L^p(\R)$$
and therefore again by \autocite[Theorem~3.14 on \pno~69]{papaRudin}, $\bigcap_{1\le p < \infty} L^p(\R)$ is dense in $L^q(\R)$ for all $1 \le q < \infty$.
